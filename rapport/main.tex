\documentclass[french]{article}
\usepackage{makecell}

\input{configuration}
\input{titre}


\begin{document}
	\myTitle[Rapport d'exécutifs temps-réel \\ - Transfert de messages - ]
	\newpage
	
	\tableofcontents
	\newpage
	\listoffigures
	\listoftables
	\newpage
	
	
	\vspace*{4cm}
	\section*{Résumé}
	L'exécutif temps-réel de systèmes électroniques est une discipline essentielle lorsqu'il s'agit, dans le cadre d'une future carrière d'ingénieur, de réaliser un produit répondant à des contraintes de temps. Et ce domaine est d'autant plus indispensable dans un monde où les systèmes numériques sont de plus en plus complexes et rapides, demandant de réaliser des tâches dans une fenêtre de temps bien déterminé. Ce rapport s'inscrit dans la présentation d'une application affichant l'heure sur un écran.

	
	\newpage
	\pagestyle{plain} % Début de la numérotation des pages
	
	\section{Introduction}
	\subsection{Contextualisation}
	La formation ETN (Électronique et technologies numériques) offerte par l'école polytechnique de l'Université de Nantes propose d'aborder diverses branches de l'électronique, du traitement du signal au systèmes à microprocesseur en passant par l'électronique analogique des hautes-fréquences. Cet ensemble de domaines techniques nécessite des compétences en matière de méthodologie de conception. Ce rapport s'inscrit dans la conception d'un appareil de marquage routier avec la méthode MCSE. La méthode MCSE (Méthode de conception des systèmes électroniques), née à Ireste par l'impulsion de Jean-Paul Calvez, cette méthode a été implantée au sein d'un outil nommée CoFluent rachetée par Intel\mbox{\textregistered } depuis 2011. Cette méthode fait désormais partie de la culture de la formation et constitue l'outil de conception premier de l'ingénieur ETN.\\
	Ce rapport se décompose en diverses parties. Il s'agira dans un premier temps de rappeler le cahier des charges de la conception de cet appareil de marquage routier. Dans un second temps la partie spécification sera traitée et pour finir il s'agira de parler de la conception.\\
	
	
	\subsection{Objectifs du rapport}
	Ce rapport vise à 
	
	\newpage
	
	

	
\end{document}
